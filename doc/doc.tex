\documentclass{article}
\usepackage[utf8]{inputenc}

\usepackage{fullpage}
\usepackage{amsmath}
\usepackage{amsfonts}

\title{Numerical Methods for Solving Navier-Stokes Equations}
\author{Nikolas Tselepidis}
\date{}

\begin{document}

\maketitle

\section{Finite Pr Equations}

\paragraph{Conservation of momentum (dimensionless)}

\begin{equation}
\frac{1}{Pr} \left( \frac{\partial \vec{v}}{\partial t} + \vec{v} \cdot \nabla \vec{v} \right) =
    - \nabla P + \nabla^2 \vec{v} + Ra T \hat{y}
\end{equation}

Taking the curl of equation (1), and substituting the vorticity $\vec{\omega} = \nabla \times \vec{v}$, it follows that:

\begin{equation}
    \frac{1}{Pr} \left( \frac{\partial \omega}{\partial t} + v_x \frac{\partial \omega}{\partial x} + v_y \frac{\partial \omega}{\partial y} \right) =
     \nabla^2 \omega - Ra \frac{\partial T}{\partial x}
\end{equation}

In 2D only one component of vorticity is needed, i.e. the one perpedicular to the 2D plane:  $\nabla^2 \psi = \omega_z$.

\paragraph{Conservation of mass (continuity)}
\begin{equation}
    \frac{\partial T}{\partial t} + \vec{v} \cdot \nabla T = \kappa \nabla^2 T
\end{equation}

\paragraph{Conservation of energy}
\begin{equation}
    \nabla \cdot \vec{v} = 0
\end{equation}


\section{Solve}

\begin{equation}
    \nabla^2 \psi = \omega
\end{equation}

\begin{equation}
    \left( v_x, v_y \right) = \left( \frac{\partial \psi}{\partial y}, -\frac{\partial \psi}{\partial x} \right)
\end{equation}

\begin{equation}
    \frac{\partial T}{\partial t} = -v_x \frac{\partial T}{\partial x} -v_y \frac{\partial T}{\partial y} + \kappa \nabla^2 T
\end{equation}

\begin{equation}
    \frac{\partial \omega}{\partial t} = -v_x \frac{\partial \omega}{\partial x} -v_y \frac{\partial \omega}{\partial y} + Pr \nabla^2 \omega - Ra Pr \frac{\partial T}{\partial x}
\end{equation}

\end{document}
